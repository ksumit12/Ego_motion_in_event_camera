\chapter{Problem Formulation}\label{chap:problem}

\section{Notation and Event Stream Model}
Let $\Omega \subset \mathbb{Z}^2$ denote the pixel grid and $t \in \mathbb{R}_{\ge 0}$ denote continuous time. An event is the tuple
\[
e_i := (x_i, y_i, t_i, p_i) \in \Omega \times \mathbb{R}_{\ge 0} \times \{-1, +1\},
\]
with polarity $p_i$ indicating the sign of the log-intensity change. The event stream over an interval $[0,T]$ is the multiset $E := \{e_i\}_{i=1}^{N(T)}$.

We adopt the standard log-intensity threshold model of event generation: a pixel triggers an event when the change in log-intensity exceeds a (per-pixel) contrast threshold $c$; see \cite{Lichtsteiner2008DVS,Brandli2014DAVIS,Posch2014Retinomorphic,Gallego2020Survey}. Following common practice, the stream can be represented as a distribution of impulses at timestamps and locations; e.g., an event field $E(p,t)$ integrates Dirac deltas whose amplitudes encode polarity and quantized contrast \cite{Scheerlinck2021Thesis,Wang2025Thesis}:
\begin{equation}
E(p,t) = \sum_{i} \sigma_i\,c\,\delta\!\left(t - t_i\right)\,\mathbf{1}\{p=(x_i,y_i)\}, \qquad \sigma_i \in \{-1,+1\}.
\label{eq:event-field}
\end{equation}
Per-pixel non-idealities (background activity, polarity asymmetry, threshold mismatch, refractory effects) are acknowledged and handled operationally through robust matching tolerances and polarity checks introduced below \cite{Brandli2014DAVIS,Delbruck2020Handbook,Gallego2020Survey}. Timestamp jitter on modern devices is typically small relative to other uncertainties at sub-millisecond horizons \cite{Wang2025Thesis}.

\paragraph{Goal.}
Given an event stream $E$ from a static scene under \emph{rotational ego-motion}, we seek to identify and \emph{cancel} events predictable from a circular-motion model, \emph{asynchronously and per-event}. The aim is to reduce ego-motion clutter early, preserving low latency while leaving residual events that more likely correspond to independent scene/object motion.

\section{Circular-Motion Ego-Motion Model}
Over short horizons, we model the dominant camera motion as planar rotation about an image-plane center $c=(c_x,c_y) \in \mathbb{R}^2$ with signed angular velocity $\omega \in \mathbb{R}$. For $x \in \mathbb{R}^2$, define the rotation of $x$ about $c$ by angle $\theta$ as
\begin{equation}
\mathcal{R}(x; c, \theta) \;=\; c \;+\; R(\theta)\,(x-c),
\qquad
R(\theta) \;=\; 
\begin{bmatrix}\cos \theta & -\sin \theta\\ \sin \theta & \cos \theta\end{bmatrix}.
\label{eq:rot-operator}
\end{equation}
Assuming $\omega$ locally constant over a short prediction horizon $\Delta t$, the \emph{future} location of $x$ is
\begin{equation}
x' \;=\; \mathcal{R}\!\left(x;\, c, \omega\,\Delta t\right).
\label{eq:forward-prop}
\end{equation}
For an event $e_i=(x_i,y_i,t_i,p_i)$ at time $t_i$, write $x_i=(x_i,y_i)$ and denote its forward-predicted location at time $t_i+\Delta t$ as $x_i' = \mathcal{R}(x_i; c, \omega\,\Delta t)$.

\paragraph{Rationale and scope.}
Circular motion is analytically compact (parameters $(c,\omega)$), commonly used for evaluation (spinning discs/wheels), produces dense stress-test patterns, and admits accurate rotation-only estimation \cite{Gallego2017Angular,Stoffregen2019Segmentation,Gallego2018CMax}. Over short horizons, translation is small compared to rotation-induced flow for our apparatus and is neglected to first order (we revisit this in \S\ref{sec:assumptions}).

\section{Per-Event Forward Prediction with a Temporal Gate}\label{sec:temporal_gate}
    We adopt a causal \emph{predict–wait–match} strategy with an explicit temporal tolerance:

\begin{enumerate}
  \item \textbf{Predict.} Upon observing $e_i$ at $t_i$, compute $x_i'=\mathcal{R}(x_i; c, \omega\,\Delta t)$ and the decision time $t^* = t_i + \Delta t$ for the predicted future location $x_i'$ with predicted opposite polarity $\bar p_i=-p_i$ (see polarity discussion below).
  \item \textbf{Wait.} Defer decision until the wall-clock reaches the decision time $t^*$.
  \item \textbf{Temporal gate.} At $t^*$, consider only \emph{real} events whose timestamps lie inside a symmetric window:
  \begin{equation}
  \mathcal{N}_t(t^*;\epsilon_t) \;=\; \{\, e_j=(x_j,y_j,t_j,p_j)\in E \,:\, |t_j - t^*| \le \epsilon_t \,\}.
  \label{eq:temporal-window}
  \end{equation}
  \item \textbf{Spatial/polarity gate.} Within $\mathcal{N}_t(t^*;\epsilon_t)$, search for $e_j$ satisfying
  \begin{equation}
  \|x_j - x_i'\|_2 \le \epsilon_{xy},
  \qquad
  \pi(p_i,p_j)=1,
  \label{eq:gating}
  \end{equation}
  where $\epsilon_{xy}>0$ is a spatial tolerance and $\pi$ encodes the polarity rule (default: strict opposite polarity, $\pi(p_i,p_j)=\mathbf{1}\{p_j=-p_i\}$).
\end{enumerate}

If such a match exists, declare $e_i$ \emph{ego-motion predictable} and add $e_i$ to the cancel set; remove both $e_i$ and the matched event $e_j$ from the stream (software-level cancellation). If no match is found inside the spatio-temporal gate, $e_i$ remains in the residual set.

\paragraph{Remarks on implementation.}
Equation~\eqref{eq:temporal-window} enforces a temporal gate with tolerance $\epsilon_t$ (no fixed time bins). In practice, we maintain two time-sorted buffers and advance sliding indices so that, for each $t^*$, we retrieve exactly those real events with $|t - t^*| \le \epsilon_t$ before applying the spatial/polarity test~\eqref{eq:gating}. This avoids boundary artifacts inherent to fixed-bin discretization and matches the theory exactly. The cancellation is implemented by removing matched pairs from the event stream, preserving causality and low latency without hardware-level interference.

\paragraph{Discussion of polarity.}
Over short intervals, a moving edge driving monotonic log-intensity change tends to produce consistent polarity. We therefore default to \emph{opposite-polarity cancellation} (matching predicted opposite polarity $\bar p_i$ with the real event polarity). When polarity asymmetries are suspected (sensor mismatch), we also evaluate a relaxed polarity policy in sensitivity analyses.

\subsection{Matching Policy and One-to-One Constraints}
Let $\mathcal{N}_t(t^*;\epsilon_t)$ denote the candidate window at decision time. We impose a \emph{one-to-one} pairing between predicted and real events within that window to prevent over-cancellation. A simple and effective policy is \emph{mutual nearest neighbors} within the spatial tolerance: for each real event, find its nearest predicted candidate and vice versa; accept a pair only if both choose each other and~\eqref{eq:gating} holds (polarity included). Greedy distance-ordered matching with a ``used set'' for predicted events is an alternative with similar behavior.

\subsection{Sets, Counters, and Residuals}
Over $[0,T]$, let $E$ denote all observed events, $\mathcal{C}\subseteq E$ the subset declared cancellable under \eqref{eq:temporal-window}--\eqref{eq:gating} and removed via software-level cancellation, and $\mathcal{R} := E \setminus \mathcal{C}$ the \emph{residual} events. Define the \emph{cancellation ratio}
\begin{equation}
\mathrm{CR} \;:=\; \frac{|\mathcal{C}|}{|E|},
\qquad
\mathrm{RR}_\mathcal{A} \;:=\; \frac{|\mathcal{R}\cap \mathcal{A}|}{|\mathcal{A}|}
\label{eq:cr-rr}
\end{equation}
for a region-of-interest $\mathcal{A}\subseteq \Omega\times[0,T]$ (e.g., disc vs.\ background masks). We also report \emph{residual event density} $\rho_\mathcal{A} := \frac{|\mathcal{R}\cap (\Omega_\mathcal{A}\times[0,T])|}{|\Omega_\mathcal{A}|}$ and radial residual profiles on the disc (\S\ref{sec:metrics-bridge}).

\subsection{Causality, Buffers, and Complexity}
The per-event pipeline maintains short spatio-temporal buffers: (i) a small queue of pending predictions indexed by decision times $t^*$; (ii) a time-sorted buffer of real events for the sliding temporal window \eqref{eq:temporal-window}; and (iii) a light spatial index (e.g., grid or k-d tree) for neighborhood queries within radius $\epsilon_{xy}$. All decisions at $t^*$ use only past and present data, preserving causality and low latency (no global batch optimization), similar in spirit to asynchronous processing emphasized in \cite{Wang2025Thesis,Scheerlinck2021Thesis}.

\section{Geometric and Temporal Sensitivity to Model Errors}
Prediction accuracy depends on parameter bias and timing. Let the true parameters be $(c^\star,\omega^\star)$ and the estimates be $(\hat c,\hat\omega)$. Consider an event at radius $r=\|x-c^\star\|_2$ from the true center.

\paragraph{Angular-velocity bias.}
Let $\Delta\omega := \hat\omega-\omega^\star$. Over horizon $\Delta t$, the true angular displacement is $\theta^\star=\omega^\star \Delta t$ and the predicted is $\hat\theta=\hat\omega \Delta t$. The angular error $\delta\theta=\hat\theta-\theta^\star=(\Delta\omega)\Delta t$ induces a spatial prediction error
\begin{equation}
\varepsilon_{\omega}(r,\Delta t) \;=\; \big\|\mathcal{R}(x;c^\star,\hat\theta)-\mathcal{R}(x;c^\star,\theta^\star)\big\|
\;=\; 2r\,\big|\sin(\delta\theta/2)\big|
\;\approx\; r\,|\Delta\omega|\,\Delta t,
\label{eq:angvel-error}
\end{equation}
where the linear approximation holds for small $\delta\theta$.

\paragraph{Center-of-rotation bias.}
Let $\Delta c := \hat c - c^\star$. For small $\Delta t$, the first-order contribution of $\Delta c$ to the spatial error satisfies
\begin{equation}
\varepsilon_{c}(r) \;\lesssim\; \|\Delta c\|_2,
\label{eq:center-error}
\end{equation}
and couples with (\ref{eq:angvel-error}) for general $r$.

\paragraph{Timing uncertainty, horizon, and the temporal gate.}
Let $\sigma_t$ denote effective timing uncertainty (sensor latency variability, timestamp quantization; see hardware timing notes in \cite{Lichtsteiner2008,Delbruck2014ISCAS}). Over horizon $\Delta t$, phase uncertainty is $\sigma_\theta \approx |\omega^\star|\,\sigma_t$, contributing spatial uncertainty $\sigma_x \approx r\,\sigma_\theta \approx r\,|\omega^\star|\,\sigma_t$. The temporal gate \eqref{eq:temporal-window} explicitly allows $|t_j-t^*|\le \epsilon_t$ so that modest timing error does not cause missed matches. However, increasing $\Delta t$ amplifies phase error via~\eqref{eq:angvel-error}; thus $\Delta t$ trades discriminability (separation in time) against phase robustness.

\paragraph{Implications for the spatial gate.}
A sufficient condition for a match is
\begin{equation}
\varepsilon_{\omega}(r,\Delta t) + \varepsilon_c(r) + \sigma_x \;\le\; \epsilon_{xy},
\label{eq:gate-condition}
\end{equation}
which explains typical cancellation curves: $\mathrm{CR}$ increases with $\epsilon_{xy}$ but saturates/degrades if $\epsilon_{xy}$ is too large (false pairs); and $\mathrm{CR}$ decreases with large $\Delta t$ as phase error grows. These trends motivate the parameter sweeps in Chapter~\ref{chap:metrics}.

\section{Formal Cancellation Rule and Objective}
Define the decision indicator
\begin{equation}
\delta(e_i; \hat c,\hat\omega,\Delta t,\epsilon_{xy},\epsilon_t) \;=\;
\begin{cases}
1, & \text{if } \exists\, e_j \in \mathcal{N}_t(t_i+\Delta t;\epsilon_t) \text{ s.t. } \eqref{eq:gating}\text{ holds},\\
0, & \text{otherwise.}
\end{cases}
\label{eq:delta-indicator}
\end{equation}
Then $\mathcal{C}=\{e_i\in E:\delta(e_i;\cdot)=1\}$ and $\mathcal{R}=E\setminus \mathcal{C}$. Operationally we aim to maximize cancellation while avoiding over-cancellation:
\begin{align}
\max_{\hat c,\hat\omega,\Delta t,\epsilon_{xy},\epsilon_t} \quad & \mathrm{CR} \;-\; \lambda\, \mathrm{OC}, 
\label{eq:obj}\\
\text{s.t.} \quad & \text{causality and bounded memory},
\nonumber
\end{align}
where $\mathrm{OC}$ is an over-cancellation proxy (e.g., fraction of cancellations outside the disc ROI or violating polarity), and $\lambda\ge 0$ trades cancellation against safety. We do not solve \eqref{eq:obj} globally; instead, Chapter~\ref{chap:metrics} studies the \emph{sensitivity} of outcomes to $\Delta t,\epsilon_{xy},\epsilon_t$ given $(\hat c,\hat\omega)$.

\section{Cancellation Implementation and Representation}
On a successful match, we remove both the originating event $e_i$ and the matched event $e_j$ from the event stream, effectively suppressing the predicted observation in the maintained event representation (e.g., polarity-separated accumulation or voxel grid). This software-level cancellation operates at the decision time $t^*$ with microsecond-scale delay, consistent with asynchronous processing goals \cite{Gallego2018CMax,Bardow2016SOFIE}. In evaluation, matched pairs are simply omitted from downstream processing, avoiding the need for explicit ``anti-event'' data structures.

\section{Assumptions, Limitations, and Threats to Validity}
\label{sec:assumptions}
\textbf{(A1) Static scene.} The background is static; independently moving objects are not modeled by the circular ego-motion and should remain as residuals.

\textbf{(A2) Short-horizon rotation dominance.} Over small $\Delta t$, rotational flow dominates translation for our apparatus; accelerations are small over the horizon. This aligns with rotation-focused practice in event tracking and compensation \cite{Gallego2017Angular,Gallego2018CMax}.

\textbf{(A3) Calibration and warping.} Intrinsics and distortion are pre-compensated or negligible on the ROI; otherwise, normalize coordinates before applying~\eqref{eq:forward-prop}.

\textbf{(A4) Sensor non-idealities.} Background activity, threshold mismatch, polarity asymmetry, and refractory effects are present \cite{Brandli2014DAVIS,Delbruck2020Handbook,Gallego2020Survey}. We mitigate via strict polarity checks, tuned tolerances $(\epsilon_{xy},\epsilon_t)$, and background baselines in the metrics. Timestamp jitter is comparatively minor at our horizons \cite{Wang2025Thesis}.

\textbf{(A5) Causality and bounded memory.} Decisions use only past/present data in short buffers; we avoid global contrast-maximization windows and maintain per-event latency \cite{Bardow2016SOFIE,Gallego2018CMax}.

\paragraph{Failure modes.}
(i) Larger $\Delta t$ and/or angular-velocity bias $\Delta\omega$ cause phase errors (\ref{eq:angvel-error}) and missed matches; (ii) miscentered rotation $\Delta c$ yields radius-dependent residuals; (iii) overly large tolerances over-cancel by matching unrelated events; (iv) flicker can pass spatio-temporal gates unless filtered; (v) pronounced translation or non-circular motion violates the model.

\section{Bridge to Metrics and Experiments}
\label{sec:metrics-bridge}
We quantify performance via: (i) cancellation ratio $\mathrm{CR}$ \eqref{eq:cr-rr}; (ii) residual density $\rho_\mathcal{A}$ on disc vs.\ background masks; (iii) radial residual profiles on the disc; and (iv) sensitivity curves vs.\ $\Delta t$, $\epsilon_{xy}$, $\epsilon_t$, polarity handling, and parameter bias (Chapter~\ref{chap:metrics}). These choices reflect established motion-compensation practice (IWE/contrast) and event-rate diagnostics \cite{Bardow2016SOFIE,Gallego2018CMax,Stoffregen2019Segmentation,Gallego2020Survey}, adapted to proactive per-event cancellation.
