\chapter{Experimental Setup}
\label{chap:setup}

This chapter documents the hardware apparatus, data-collection protocol, and software pipeline used to evaluate the causal, per-event cancellation method described in Chapters~\ref{chap:problem}--\ref{chap:cancellation}. Our goal is to provide enough detail for faithful reproduction: sensor and rig configuration, calibration steps, sequence design, and the end-to-end processing stack (estimation $\rightarrow$ prediction $\rightarrow$ gating $\rightarrow$ metrics).

% ============================================================
\section{Hardware and Apparatus}
\label{sec:hardware}

\subsection{Event Camera and Optics}
We used a single event-based sensor mounted on a rigid aluminum frame facing a rotating disc (Fig.~\ref{fig:rig}). The relevant characteristics:

\begin{itemize}
  \item \textbf{Sensor:} [Specify your sensor model here, e.g., DVS128, DAVIS346, or other]. Resolution: 1280×720 pixels. Timestamp precision on the order of microseconds, polarity encoding in $\{\pm1\}$ (or $\{0,1\}$). Dynamic range exceeding 120\,dB.
  \item \textbf{Lens:} Standard C-mount lens, focal length selected to provide adequate field of view for the disc. Focus was carefully set at the disc plane to maximize edge sharpness and contrast. Aperture chosen to avoid saturation while maintaining high contrast on the textured disc patterns.
  \item \textbf{Mounting:} Camera clamped securely to aluminum extrusions to minimize vibration and mechanical drift. The baseline between the optical axis and disc center was kept approximately orthogonal to the disc plane to reduce perspective distortion over the region of interest. Test equipment (oscilloscope, signal generator) visible in Fig.~\ref{fig:rig} supported synchronization and timing measurements.
  \item \textbf{Working distance:} Approximately 60--80\,cm from sensor to disc plane (calibrated via known disc diameter and imaged radius), yielding an imaged disc radius of 264\,px at the chosen focal length. This provided sufficient spatial resolution for motion estimation and cancellation analysis.
\end{itemize}

\subsection{Spinning Disc Rig}
The target is a flat wooden disc with an \emph{uneven, high-contrast pattern} adhered to the visible face to generate rich edge structures under rotation (Fig.~\ref{fig:disc}). Hardware elements:

\begin{itemize}
  \item \textbf{Disc:} Circular wooden plate approximately 30--40\,cm in diameter, with thickness sufficient to maintain rigidity under rotation. Six square paper patterns are affixed to the disc face using blue tape at the corners, creating a grid layout (visible in Fig.~\ref{fig:rig}). The patterns consist of:
  \begin{itemize}
    \item Four squares with intricate, flowing concentric black and white lines resembling topographical contours or fingerprint patterns, providing dense, curved edge structures.
    \item Two squares (top corners) with dense random assortments of black and white shapes including squiggles, hearts, stars, circles, and crosses, creating high-frequency spatial content.
  \end{itemize}
  These patterns generate a rich variety of edge orientations and spatial frequencies to stress-test the cancellation algorithm across different geometric configurations.
  \item \textbf{Actuation:} Precision DC or BLDC motor with controller; rotation rate adjustable via voltage or PWM command from the test equipment (visible on top shelf in Fig.~\ref{fig:rig}). Nominal speeds tested: 1.5--3.5\,rev/s, corresponding to angular velocities $\omega \approx 1.5--3.5$\,rad/s, with steady-state angular velocity ripple below 2\% after warm-up.
  \item \textbf{Spindle and bearings:} Low-runout precision bearings support the spindle to minimize lateral wobble and eccentricity. A hub mount ensures concentricity between the disc and rotation axis, maintaining stable circular motion for consistent experimental conditions.
  \item \textbf{Coordinate system:} A red hand-drawn coordinate system on the grey mounting board (visible in Fig.~\ref{fig:rig}, lower-left) defines the reference frame with $x$ pointing right and $y$ pointing down, consistent with image-plane conventions.
\end{itemize}

\subsection{Lighting Conditions}
Event generation depends on brightness gradients and texture. We used diffuse LED panels positioned off-axis to minimize specular highlights on the disc. Ambient flicker was avoided by using DC-powered lights. Illumination was kept constant throughout each sequence; no HDR sweeps were performed. Before acquisition, we verified that background-activity noise (on a static scene) remained low relative to disc-induced events.

\subsection{Calibration and Referencing}
We performed lightweight calibration sufficient for short-horizon rotation modeling:

\begin{itemize}
  \item \textbf{Intrinsics/distortion:} Intrinsics were either pre-calibrated or assumed approximately linear over the disc region. If distortion was noticeable, we undistorted $(x,y)$ prior to prediction (see Chapter~\ref{chap:motion}).
  \item \textbf{Disc center in image:} A quick \emph{circle fit} (Pratt/Taubin) on a short burst of disc events estimated $(c_x,c_y)$; the estimate was compared with any external tracker CSV center (when available). The operational center $\hat c$ used for prediction was the smoothed (EMA) time series.
  \item \textbf{Angular velocity reference:} Angular position $\theta(t)$ was derived from the event cluster around the rim (or from tracker CSV) and differentiated to obtain $\omega(t)$, then smoothed (EMA) to reduce jitter. These trajectories form the time-aligned $(\hat c(t),\hat\omega(t))$ used in forward prediction.
\end{itemize}

\begin{figure}[t]
  \centering
  \includegraphics[width=0.85\linewidth]{../images/setup_2.jpeg}
  \caption{Experimental apparatus. The event camera (not visible, positioned off-camera right) views a patterned wooden disc mounted on a precision rotating spindle. The disc features six high-contrast paper patterns affixed with blue tape, creating varied edge structures for rich event generation. The grey mounting board provides a stable backdrop with a hand-drawn coordinate system (red $x$/$y$ axes, lower-left). Test equipment on the top shelf (oscilloscope, signal generator) supports timing and synchronization measurements. The aluminum extrusion frame ensures rigidity and controlled motion.}
  \label{fig:rig}
\end{figure}

\begin{figure}[t]
  \centering
  \includegraphics[width=0.7\linewidth]{../images/setup_1.jpeg}
  \caption{Close-up view of the patterned disc face. Six square paper patterns are arranged in a grid, secured with blue tape at the corners. Four patterns (center and lower rows) feature flowing concentric black and white lines creating curved edge structures. Two patterns (top corners) contain dense random assortments of shapes (squiggles, stars, circles, hearts) for high-frequency spatial content. This variety of textures ensures comprehensive testing of the cancellation algorithm under different edge orientations and densities.}
  \label{fig:disc}
\end{figure}

% ============================================================
\section{Datasets and Protocol}
\label{sec:datasets}

\subsection{Sequences}
We recorded multiple \emph{spinning-disc} sequences by varying the rotation rate and exposure to stress different motion regimes. Each sequence contains the raw, time-ordered event stream:
\[
E = \big\{ (x_i, y_i, p_i, t_i) \big\}_{i=1}^{N},
\]
with timestamps in microseconds (converted to seconds in software). For most analyses we also used a tracker CSV with center and angular-velocity estimates; when absent, these were inferred from events via circle fit and angle differencing (Chapter~\ref{chap:motion}).

\paragraph{Typical settings.}
We recorded sequences of 5--10\,s duration per capture, at rotation rates spanning 1.5--3.5\,rev/s ($\omega \approx 1.5--3.5$\,rad/s), to study the dependence of cancellation on angular velocity $\omega$ and radial phase speed $r|\omega|$. For the disc radius $r \approx 264$\,px imaged at this setup, the maximum circumferential velocity at the rim reached $\sim 800$\,px/s, creating high apparent motion to stress-test the temporal gating mechanism. Lighting and camera pose remained fixed across repeats within a set to ensure consistency.

\subsection{Protocols and Repeats}
To assess repeatability, each setting was recorded at least [\textit{$n$}] times. For sensitivity experiments we held all parameters fixed and varied one control at a time (e.g., $\Delta t$, $\epsilon_{xy}$, $\epsilon_t$, polarity mode, $\omega$-bias). Each sweep used identical temporal extents (three non-overlapping windows per sequence) to enable cross-window comparison.

\subsection{Data Splits}
No learning was employed; thus no train/validation/test split was required. Instead, we designate \emph{analysis windows} (e.g., W1, W2, W3) within each sequence and report per-window curves (cancellation vs.\ parameter) as well as their mean and standard deviation. When we used tracker CSV for $(\hat c,\hat\omega)$, we withheld a short pre-roll window to tune smoothing constants and any constant bias correction.

\subsection{Summary Table}
Table~\ref{tab:sequences} summarizes the datasets used in the study. Replace placeholders with your actual counts.

\begin{table}[H]
  \centering
  \caption{Sequences used in experiments. ``Rate'' is the nominal spin frequency. $N$ is total events in the sequence. Analysis windows W1--W3 are non-overlapping 10\,ms slices used for robustness checks.}
  \label{tab:sequences}
  \begin{tabular}{lccccl}
    \toprule
    \textbf{Seq.} & \textbf{Duration (s)} & \textbf{Rate (rad/s)} & \textbf{Windows} & \textbf{$N$ (events)} & \textbf{Purpose} \\
    \midrule
    Perlin-1280Hz & 10.0 & $\sim 2.0$ & W1--W6 & $\sim 36.3 \times 10^6$ & Primary analysis \\
    \bottomrule
  \end{tabular}
\end{table}

\paragraph{Data characteristics.}
The dataset comprises a single continuous 10-second recording at 1280\,Hz effective temporal sampling from the event sensor. The sequence contains three distinct high-contrast patterns (visible in Fig.~\ref{fig:disc}), creating varied edge densities across the disc face. Analysis was performed on six non-overlapping 10\,ms windows (W1--W6) extracted at $t = 5.0$, 5.01, 8.2$, 8.21$, 9.0$, and 9.01$\,s, enabling statistics across temporal variations in motion estimation and event density.

% ============================================================
\section{Software Pipeline}
\label{sec:software}

\subsection{Overview}
Figure~\ref{fig:pipeline} outlines the causal, per-event pipeline:

\begin{enumerate}
  \item \textbf{I/O \& preprocessing:} Load events CSV/NPY; map polarities to $\{0,1\}$ or $\{\pm 1\}$; convert timestamps to seconds; time-sort (if needed).
  \item \textbf{Motion estimation:} Obtain $(\hat c(t),\hat\omega(t))$ from tracker CSV or compute from events (circle fit $\rightarrow$ angles $\rightarrow$ finite differences), then smooth and store as time series.
  \item \textbf{Event prediction:} For each event $e_i=(x_i,t_i,p_i)$, compute $x_i' = \mathcal{R}\!\big(x_i; \hat c(t_i), \hat\omega(t_i)\Delta t\big)$ and decision time $t_i' = t_i+\Delta t$.
  \item \textbf{Temporal gating:} At wall-clock $\tau$, pop predictions with $|t_i' - \tau| \le \epsilon_t$ and gather real events $e_j$ with $|t_j - t_i'| \le \epsilon_t$.
  \item \textbf{Spatial \& polarity gating:} Accept candidates with $\|x_j - x_i'\|_2 \le \epsilon_{xy}$ and polarity predicate satisfied (default: opposite).
  \item \textbf{One-to-one pairing:} Mutual-nearest-neighbor (MNN) check within the gated sets; greedily accept pairs in ascending distance to prevent many-to-one matches.
  \item \textbf{Cancellation / anti-event:} Remove matched pairs from the stream (or emit an anti-event at $(x_i',t_i',-p_i)$ in a signed raster accumulator).
  \item \textbf{Metrics \& visualization:} Compute cancellation ratio, residual densities (disc vs.\ background), and generate high-pass videos via bilinear splatting.
\end{enumerate}

\begin{figure}[t]
  \centering
  % \includegraphics[width=0.95\linewidth]{figures/pipeline_placeholder.pdf}
  \caption{Software pipeline: estimation $\rightarrow$ prediction $\rightarrow$ temporal gate $\rightarrow$ spatial/polarity gate $\rightarrow$ one-to-one pairing $\rightarrow$ cancellation $\rightarrow$ metrics. All decisions are causal with bounded buffers.}
  \label{fig:pipeline}
\end{figure}

\subsection{Implementation Details}
We implemented the pipeline in Python:
\begin{itemize}
    \item \textbf{Environment:} Python~3.8+, NumPy (1.21+), SciPy (cKDTree for spatial indexing), pandas (data analysis), matplotlib/seaborn (visualization), opencv-python (optional: additional image processing).
  \item \textbf{Precision:} Coordinates and polarity as \texttt{float32}; timestamps as \texttt{float64}. Trigonometric rotation uses radians.
  \item \textbf{Interpolation:} Linear \texttt{interp1} for $\hat c(t)$, $\hat\omega(t)$ at \emph{event times} $t_i$ with edge-hold.
  \item \textbf{Temporal buffers:} A min-heap keyed by $t'$ for predictions; a deque for recent real events; two indices maintain the $[t'-\epsilon_t,\,t'+\epsilon_t]$ window.
  \item \textbf{Spatial index:} KD-tree (SciPy) or uniform grid on gated candidates. Matching performed with MNN to enforce one-to-one.
  \item \textbf{Rasterization:} Signed bilinear splatting of residual (uncancelled) events for stable visualization; optional nearest-neighbor for ablation.
  \item \textbf{Parameters (defaults):} $\Delta t \in [0.25, 2]$\,ms, $\epsilon_t \in [0.25, 1.0]$\,ms, $\epsilon_{xy}\in [1.5, 3.0]$\,px; polarity mode = \texttt{opposite}.
\end{itemize}

\subsection{Runtime and Throughput}
On a modern workstation (e.g., Intel Core i7 or AMD Ryzen 7, 16--32\,GB RAM), processing proceeds in chunks of $10^6$ events to balance memory usage and efficiency. The predictor is embarrassingly parallel over events via vectorized NumPy operations; the dominant computational cost is spatial KD-tree querying within the temporal gate. Typical throughputs observed:

\begin{itemize}
  \item \textbf{Prediction:} $\geq 50$ million events/s (pure vectorized rotation with NumPy), limited only by memory bandwidth.
  \item \textbf{Matching (with $N=30,000$ events per temporal slice):} $\sim 5--10$ thousand matched pairs/s (cKDTree construction + nearest-neighbor queries). The bottleneck is the spatial indexing for each event, which scales as $O(N \log N)$.
  \item \textbf{Full pipeline (per $10^6$ events):} $\sim 10--20$ seconds end-to-end (I/O, estimation, prediction, matching, metrics), enabling interactive parameter sweeps on modest hardware.
  \item \textbf{Visualization:} Real-time generation of PNG frames with signed-bilinear splatting; MP4 encoding (if performed) is bounded by I/O rather than compute.
\end{itemize}

For the comprehensive parameter sweep (DT values 0--20\,ms, 5 spatial tolerances, 5 temporal tolerances = 525 combinations), total processing time on a single workstation ranged from $4--8$ hours depending on dataset size ($\sim 30$\,M events), demonstrating the feasibility of thorough sensitivity analysis.

\subsection{Reproducibility and Configuration}
All experiments are driven by a single configuration file specifying $(\Delta t,\epsilon_t,\epsilon_{xy})$, polarity mode, interpolation method, and output paths. Randomness is not used beyond the optional RANSAC in circle fitting (seed fixed). Each run stores a manifest containing code commit ID, parameter values, and checksums for input files.

\subsection{Safety Checks and Diagnostics}
We embed invariants to guard against silent errors:
\begin{itemize}
  \item Each real/predicted index can appear in at most one match.
  \item Predictions leaving the image are dropped before indexing.
  \item With exact parameters and $\Delta t\to 0$, cancellation $\to 100\%$ on synthetic rings.
  \item The measured CR vs.\ $\Delta t$ decays smoothly (no bin-edge artifacts), confirming the \emph{true temporal gate}.
  \item Residual radial profiles flag center bias: a uniform offset indicates $\|\Delta c\|$; widening with radius indicates $\Delta\omega$ drift.
\end{itemize}

\medskip
\noindent\textbf{Summary.} The apparatus isolates rotational ego-motion with a reproducible, high-contrast target, while the software stack enforces strictly causal, per-event processing. This combination allows controlled sensitivity studies over $\Delta t$, spatial/temporal tolerances, polarity handling, and model biases, which we report in Chapter~\ref{chap:metrics}.
